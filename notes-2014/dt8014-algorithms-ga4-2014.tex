\documentclass[a4paper]{article}

%\usepackage{algorithm,amsmath,amssymb,subfig,url}
\usepackage{amsmath,amssymb,booktabs}
%\usepackage[pdftex]{graphicx}
\usepackage[margin=30mm]{geometry}
%\usepackage[format=hang,labelfont=it]{caption}
%\usepackage[noend]{algorithmic}

%\floatname{algorithm}{Listing}
%\algsetup{indent=3em}
%\renewcommand{\algorithmiccomment}[1]{\quad \{\emph{#1}\}}



\begin{document}
\title{
  \large
  DT8014 Algorithms Group Activities (2014)\\
  \Large
  Week 4 -- Randomization}
\author{Roland Philippsen}
\maketitle



\noindent
Form groups of 4-5 students and work together on the following tasks.


\section*{Neighboring Solutions}

\emph{Purpose: first steps with exploring combinatorial solution spaces.}



\section*{Simulated Annealing (SA)}

\emph{Purpose: understand the basic SA iteration.}

\noindent
Consider the complete graph specified as the adjacency matrix below.
Assume you are trying to solve the TSP on this graph using SA.
The chosen transition probability function is

\begin{equation}
  P (f_i, f_j) = \begin{cases}
    1                     &\text{if}\; f_j < f_i \\
    \exp{\bigl((f_i - f_j) / T\bigr)} &\text{otherwise}
  \end{cases}
\end{equation}

\noindent
where $f_i$ is the cost of the current solution, $f_j$ the cost of the candidate solution, and $t$ is the temperature parameter.

\begin{itemize}
\item
  What is the probability of transitioning from \textbf{(A, C, B, D)} to \textbf{(A, C, D, B)} if the temperature is $T=10$?
\end{itemize}

\begin{center}
  \begin{tabular}{lrrrr}
    \toprule
      &  A &  B &  C &  D \\
    \midrule
    A &    &  8 &  5 &  6 \\
    B &  8 &    &  3 &  9 \\
    C &  5 &  3 &    &  7 \\
    D &  6 &  9 &  7 &    \\
    \bottomrule
  \end{tabular}
\end{center}

% don't forget to add the cost for closing the cycle!
%
% fi = 5+3+9+6 = 23
% fj = 5+7+9+8 = 29
%
% exp ((23 - 29) / 10) = exp (-0.6) = 0.548811636094



\section*{Genetic Algorithm (GA)}

\emph{Purpose: think about how to encode candidate solutions for GAs.}

\noindent
In the TSP, every vertex must be visited exactly once.
This is called a \emph{tour}.
In order to apply crossover, which is a fundamental components of GAs, it would thus be best to use an encoding that always corresponds to a tour.

In its basic form, the crossover operation takes two individuals from the current population, chooses a random crossover point, and swaps the encoding subsequence before and after this point.
Formally, let the sequence $(s_1 \ldots s_n)$ denote one of the individuals, and $(t_1 \ldots t_n)$ the other.
The crossover point $x$ is drawn e.g.\ from the uniform distribution, and the new sequences will be $(s_1 \ldots s_x, t_{x+1} \ldots t_n)$ and $(t_1 \ldots t_x, s_{x+1} \ldots s_n)$.

\begin{itemize}
\item
  Design an encoding, or modify the crossover operation, such the sequences are always tours.
\end{itemize}



\section*{Ant Colony Optimisation (ACO)}

\emph{Purpose: understand a fundamental building block of ACO.}



\end{document}
